\documentclass{article}
\usepackage{amsmath}

\begin{document}

% Title
\title{Lab 9}
\author{Andry Paez}
\date{}
\maketitle

\begin{section*}{Riess et al. 1998}   
    High-redshift SNe Ia are observed to be dimmer than expected in an empty
    universe (i.e., $\Omega M = 0$) with no cosmological constant. A cosmological expla-
    nation for this observation is that a positive vacuum energy density accelerates
    the expansion. Mass density in the universe exacerbates this problem, requiring
    even more vacuum energy. For a universe with ΩM = 0.2, the MLCS and
    template-fitting distances to the well-observed SNe are 0.25 and 0.28 mag farther
    on average than the prediction from $\Omega\Lambda = 0$. The average MLCS and template-
    fitting distances are still 0.18 and 0.23 mag farther than required for a 68.3%(1 σ)
    consistency for a universe with ΩM = 0.2 and without a cosmological constant.
    
\end{section*}
\end{document}