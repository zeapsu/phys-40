\documentclass{article}
\usepackage{amsmath}

\begin{document}

% Title
\title{\textbf{Lab 9}}
\author{Andry Paez}
\date{}
\maketitle

\begin{section}{Riess et al. 1998}   

\textit{High-redshift SNe Ia are observed to be dimmer than expected in an empty
universe} (i.e., $\Omega_M = 0$) \textit{with no cosmological constant.} A cosmological expla-
nation for this observation is that a positive vacuum energy density accelerates
the expansion. Mass density in the universe exacerbates this problem, requiring
even more vacuum energy. For a universe with $\Omega_M = 0.2$, the MLCS and
template-fitting distances to the well-observed SNe are 0.25 and 0.28 mag farther
on average than the prediction from $\Omega_\Lambda = 0$. The average MLCS and template-fitting distances are still 0.18 and 0.23 mag farther than required for a 68.3\%(1 $\sigma$)
consistency for a universe with $\Omega_M = 0.24$ and without a cosmological constant.
\end{section}
\newpage
\begin{section}{Higgs 1964}   

The simplest theory which exhibits this behavior is a gauge-invariant version
of a model used by Goldstone himself: Two real \footnote{In the present note the model is discussed mainly in classical terms; nothing is proved
about the quantized theory...} scalar fields $\varphi_1$, $\varphi_2$ and a real
vector field $A\mu$ interact through the Lagrangian density

\begin{equation}
    L = -\frac 1 2 (\nabla\varphi_1)^2 - \frac 1 2 (\nabla\varphi_2)^2 - V(\varphi_1^2 + \varphi_2^2) - \frac 1 4 F_{\mu\nu}F^{\mu\nu}, \tag{12.1}
\end{equation}
\\
where 

\begin{align*}
   \nabla_\mu\varphi_1 = \partial_\mu\varphi_1 - eA_\mu\varphi_2,
    \\
   \nabla_\mu\varphi_2 = \partial_\mu\varphi_2 + eA_\mu\varphi_1,
   \\
   F_{\mu\nu} = \partial_\mu A_\nu - \partial_\nu A_\mu,
\end{align*}
\\
$e$ is a dimensionless coupling constant, and the metric is taken as $- + + +$.

\end{section}
\newpage
\begin{section}{Einstein 1905}   
Letztere ist $\frac 3 2 (R/N )T$ , w\"{a}hrend man f\"{u}r die mittlere Gr\"{o}\ss e des Energiequan-tums unter Zugrundelegung der Wienschen Formel erh\"{a}lt:
\begin{align*}
\frac{\displaystyle\int_0^{\infty}\alpha\nu^3 e^{-\displaystyle\frac{\beta\nu}{T}}d\nu}{\displaystyle\int_0^{\infty}\frac N {R\beta\nu}\alpha\nu^3 e^{-\displaystyle\frac{\beta\nu}{T}}d\nu} = 3 \displaystyle\frac R N T\text{.}
\end{align*}
\end{section}
\end{document}