\documentclass{article}
\usepackage{amsmath}
\usepackage{amssymb}
\usepackage{gensymb}
\usepackage{mathtools}
\usepackage{setspace}
\usepackage{enumitem}
\usepackage{hyperref}

\onehalfspacing
\begin{document}

\newcommand{\ket}[1]{\left| #1 \right>}

\title{Quantum Weirdness: Exploring Quantum Error Correction}
\author{Andry Paez}
\date{}
\maketitle

\section*{Progress}

\subsection*{Channel Creation with IBM Quantum}
I created an IBM Quantum channel and set up an environment to execute quantum circuits. This included accessing real quantum hardware and simulators.

\subsection*{3-Bit Repetition Code Implementation}
I started the implementation of the basic 3-bit repetition code that will address bit-flip errors, the steps I need to do are:
\begin{itemize}
    \item \textbf{Encoding:} Encode a single logical qubit into three physical qubits using CNOT gates.
    \item \textbf{Detect Noise:} After using IBM's quantum hardware, identify noise, specifically pauli-X, or, bit-flip errors.
    \item \textbf{Decoding:} Used majority voting to correct errors and recover the logical state.
    \item \textbf{Testing:} Test the circuit in both noiseless (simulated) and noisy environments using Qiskit Runtime.
\end{itemize}

\subsection*{Demonstrating Gates}
I explored the properties of quantum gates, focusing on:
\begin{itemize}
    \item The \textbf{Hadamard gate}, which places a qubit into superposition.
    \item The \textbf{CNOT gate}, used for entangling qubits and constructing error correction codes.
\end{itemize}

\subsection*{Noise Models}
I originally planned to incorporate noise models in Qiskit to simulate real-world imperfections, but opted for real noise through using quantum hardware (thanks to qiskit runtime):

\section*{Challenges Encountered/Will Encounter}

\begin{itemize}
    \item Understanding the interplay between bit-flip errors and quantum superposition states proved challenging, particularly when applying error correction.
    \item The introduction of Qiskit Runtime required adjustments to the circuit execution workflow.
    \item Visualizing noise effects on the Bloch sphere and interpreting results in a meaningful way I foresee taking a significant effort.
\end{itemize}

\section*{To-Do List}

\begin{itemize}
    \item Extend the project to include \textbf{Shor’s Code} for correcting both bit-flip and phase-flip errors if there is time.
    \item Compare the performance of the 3-bit repetition code and Shor’s Code on real quantum hardware.
    \item Develop visual aids for the report and presentation, including:
    \begin{itemize}
        \item Histograms comparing error rates with and without error correction.
        \item Bloch sphere animations demonstrating error effects and recovery.
    \end{itemize}
    \item Finalize the report with detailed results, including metrics such as logical qubit recovery rates and fidelity.
\end{itemize}
\end{document}
